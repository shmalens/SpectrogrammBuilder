\documentclass[12pt, a4paper]{article}
\usepackage[T2A]{fontenc}
\usepackage[utf8]{inputenc}
\usepackage[russian]{babel}
\usepackage{graphicx}
\usepackage{geometry}
\usepackage{textpos}
\usepackage{textpos}
\usepackage{array}
\usepackage{caption}
\usepackage{amsmath}
\usepackage{fancyhdr}

\geometry{
	a4paper,
	textwidth=170mm,
	textheight=250mm,
	left=20mm,
	top=20mm,
}

\newcommand{\Faculty}{Робототехника и комплексная автоматизация}
\newcommand{\Department}{Системы автоматизированного проектирования (САПР)}

\newcommand{\TitleText}{@Название работы часть 1@}
\newcommand{\Title}{{\Huge \textbf{\TitleText}}}

\newcommand{\SubTitleText}{@Название работы часть 2 (опционально)@}
\newcommand{\SubTitle}{{\LARGE \textbf{\SubTitleText}}}

\newcommand{\FullName}{Дунайцев Александр Иванович}
\newcommand{\Author}{Дунайцев А. И}
\newcommand{\EduGroup}{РК6-54Б}
\newcommand{\TaskType}{@Тип документа@}
\newcommand{\WorkTheme}{@Тема работы@}

% Пример как добавить картину в документ
%\begin{figure}[h]
%	\centering    % Центрируем
%	\includegraphics[width=0.25\textwidth]{pic.png}
%	\caption{a nice plot} % Подписть к картинке, будет снизу
%	\label{fig:pic1} % Можно указывать ссылку на эту картинку в тексте, как \ref{fig:pic1}
%\end{figure}

\setlength{\columnsep}{1in}


\begin{document}
	% Место для верстки титульного документа
	\thispagestyle{empty}
	\begin{tabular}{m{0.15\linewidth}m{0.85\linewidth}}
	\centering
	\includegraphics[scale=0.07]{static/bmstu.pdf} &
	{\centering
	Министерство науки и высшего образования Российской Федерации
	Федеральное государственное бюджетное образовательное учреждение
	высшего образования
	
	«Московский государственный технический университет
	имени Н.Э. Баумана
	(национальный исследовательский университет)»
	
	(МГТУ им. Н.Э. Баумана)
	
    } \\
	\hline
	\multicolumn{1}{p{0.15\textwidth}}{} & \multicolumn{1}{p{0.85\textwidth}}{} \\
	\multicolumn{1}{p{0.15\textwidth}}{ФАКУЛЬТЕТ}	&	\multicolumn{1}{p{0.85\textwidth}}{\Faculty}	\\
	\multicolumn{1}{p{0.15\textwidth}}{КАФЕДРА}	&	\multicolumn{1}{p{0.85\textwidth}}{\Department}	\\
	\end{tabular}
\vfil

\begin{center}
	\Title
	
	\SubTitle
\end{center}


\vfil
\begin{center}
	\begin{tabular}{p{0.4\textwidth}p{0.4\textwidth}} 
		Студент:	& \FullName \\ 
		\hline
		Группа:	& \EduGroup \\ 
		\hline
		Тип задания:	& \TaskType \\ 
		\hline
		Тема:	& \WorkTheme \\ 
		\hline
	\end{tabular}
\end{center}

\vfil

\begin{tabular}{p{0.45\textwidth}p{0.25\textwidth}p{0.25\textwidth}} 
	\large
	Студент	&	$\underset{\text{подпись, дата}}{\underline{\hspace{0.2\textwidth}}}$ & \underline{\Author}  \\ 
	& & \\
	Преподаватель	&	$\underset{\text{подпись, дата}}{\underline{\hspace{0.2\textwidth}}}$ & $\underset{\text{Фамилия, И. О.}}{\underline{\hspace{0.2\textwidth}}}$ \\ 
\end{tabular}

\vfil
\vfil
\begin{center}
	Москва, 2021
\end{center}

%Конец титульного листа
\newpage	
	% Страница с версткой содержания
	\tableofcontents
	\newpage
	
	% Далее идут секции с текстом документа
	\section{Тригонометрические ряды Фурье.}
	\subsection{Определение ряда Фурье.}
	Ряд вида:
	\[
	f(x) = a_0 + \sum_{k=1}^{n} a_k cos(kx) + b_k sin(kx) \quad (a_k, b_k \in R),
	\]
	где $f \in L_2\left[ -\pi, \pi \right]$ называется тригонометрическим рядом Фурье.
	
	Коэфициенты ряда Фурье находятся по следующим формулам:
	
	\[
	a_0 = \frac{1}{\pi} \int_{-\pi}^{\pi} f(x) dx
	\]
	
	\[
	a_k = \frac{1}{\pi} \int_{-\pi}^{\pi} f(x) cos(kx) dx
	\]
	
	\[
	b_k = \frac{1}{\pi} \int_{-\pi}^{\pi} f(x) sin(kx) dx,
	\]
	
	Тригонометрический ряд Фурье можно записать в виде:
	
	\[
	f(x) = \sum_{k=0}^{n} a_k cos(kx) + b_k sin(kx) \quad (a_k, b_k \in R),
	\]
	
	где $f \in L_2\left[ -\pi, \pi \right]$. Такая форма записи позволяет увидеть, что коэфициент $b_0$ может быть произвольным, так как $sin0 = 0$. Эта форма записи окажется удобной при выводе экспоненциальной формы ряда Фурье.
	
	\subsection{Комплексная форма записи ряда Фурье.}
	
	Представить ряд Фурье можно в комплексной форме. Вывод данной формы необходимо начать с рассмотрения формулы Эйлера:
	
	\[
	e^{ikx} = cos(kx) + i sin(kx).
	\]
	
	Домножим эту формулу на комплексный коэфициент $\hat{a}_k \in C$.
	
	\[
	\hat{a}_k e^{ikx} = \hat{a}_k cos(kx) + i \hat{a}_k sin(kx).
	\]
	
	Разложим, в правой части равенства, комплексный коэфициент $\hat{a}_k$ на вещественную $Re(\hat{a}_k)$ и мнимую $Im(\hat{a}_k)$ части, получим:
	
	\[
	\hat{a}_k e^{ikx} = Re(\hat{a}_k) cos(kx) + i Im(\hat{a}_k) cos(kx) + i Re(\hat{a}_k) sin(kx) - Im(\hat{a}_k) sin(kx).
	\]
	
	Теперь, прибавив к обеим частям равенства комплексно сопряженное число $\hat{a}_k^*$, в правой части, по свойству сложения комплексно-сопряженных чисел, сократятся все слагаемые, содержащие  $i$, а остальные слагаеме удвоятся.
	
	\[
	\hat{a}_k e^{ikx} + \hat{a}_k^* e^{ikx} = 2Re(\hat{a}_k) cos(kx) - 2Im(\hat{a}_k) sin(kx).
	\]
	
	Суммирование обеих частей равенства по $k = 0, ..., n$, приводит к следующему выражению:
	
	\[
    \sum_{k=0}^{n}(\hat{a}_k e^{ikx} + \hat{a}_k^* e^{ikx}) = \sum_{k=0}^{n}[2Re(\hat{a}_k) cos(kx) - 2Im(\hat{a}_k) sin(kx)].
	\]
	 
	Если записать $\hat{a}_{-k} = \hat{a}_k^*$, то можно перейти к выражению предыдущего равенства к виду куда более похожему на тригонометрический ряд:
	
	\[
	\sum_{k=-n}^{n}\hat{a}_k e^{ikx} = \sum_{k=-n}^{n}[2Re(\hat{a}_k) cos(kx) - 2Im(\hat{a}_k) sin(kx)]. \qquad (1)
	\]
	
	Очевидна теперь и связь коэфициентов $a_k$ и $b_k$.
	
	\[
	a_k = 2 Re(\hat{a}_k),
	\]
	
	\[
	b_k = -2 Im(\hat{a}_k),
	\]
 	
 	где $k = 0, ..., n$.
 	
 	Тригонометрический ряд игногда называют тригономитрическим полиномом. Это название мотивировано экспоненциальной формой записи. Действительно, если записать $e^{ikx} = z^k$, тогда можно перейти к форме полинома.
 	
 	\[
 	\sum_{k=-n}^{n}\hat{a}_k e^{ikx} = \sum_{k=-n}^{n}\hat{a}_k z^k.
 	\]
 	
 	\section{Дискретное преобразование Фурье.}
 	Тригонометрическими полиномами можно приближать как непрерывные, так и дискретные данные.
 	
 	\subsection{Лемма о сумме триногометрического ряда.}
 	
 	
	\newpage
	\section{Литература}
	% https://mipt.ru/education/chair/mathematics/upload/8ec/besov_f-arph0dsdzp7.pdf
	
\end{document}